\documentclass{article}

\usepackage{booktabs}
\usepackage[margin=1in]{geometry}

\title{Evidencia 1. Resaltador de sintaxis}
\author{Fausto Jiménez de la Cuesta Vallejo A01027983\\
        Miguel Enrique Soria A01028033\\
        José Antonio González Martínez A01028517}
\date{\today}

\begin{document}
\maketitle

\section{Introducción}
En esta entrega se desarrollo un resaaltador de sintáxis básico en el lenguaje 
de programación Python. La herramienta permite identificar con mayor facilidad 
las palabras reservadas, operadores, variables, comentarios, etc. El 
resaltador desarrollado en python toma un archivo \texttt{.py} y lo analiza, 
generando un archivo \texttt{html} con el código resaltado.

\section{Solución}
Para realizar el resaltador, se creó un Autómata Finito (AF) que tiene como 
estados finales los grupos de tokens que llevan un mismo color para generar 
una tabla de transiciones que pueda ser implementada en un script de Pyhton. 
En nuestro caso, se utilizó un \texttt{while} loop que a través de el 
contenido del archivo y el estado en la tabla de estados revisa caracter por 
caracter el contenido del archivo hasta encontrar un grupo de caracteres que 
entre dentro de uno de los campos definidos. Utilizando la misma tabla se 
consigue el estado en el que se encuentra comparando el estado en una posicion 
dada por el valor \texttt{col} y estado para colorear apropiadamente cada 
grupo. Todo esto se engloba en la función \texttt{lexer(filename)} que toma 
como parametro el nombre del archivo \texttt{.py} a analizar como ruta relativa.

\begin{table}
\centering
\caption{Tabla de transiciones}
\label{tab:tabla}
\begin{tabular}{l l l l l l l l l l l l l l l}
    \toprule
    digito	& punto	& letra	& operador	& smbolos especiales	 & '	& " 	& \#	& e/E	& /N	& SPACE \verb|\t|	& Otro \\
    \midrule
    1	& 10	& 5	& 13	& 15	& 7	& 8	& 6	& 5	& 9	& 9	& 19 \\
    1	& 2	& 10	& 10	& 10	& 10	& 10	& 10	& 10	& 10	& 10	& 10 \\
    2	& 19	& 19	& 12	& 12	& 19	& 19	& 19	& 3	& 12	& 12	& 19 \\
    4	& 19	& 19	& 4	& 19	& 19	& 19	& 19	& 19	& 19	& 19	& 19 \\
    4	& 19	& 19	& 12	& 19	& 19	& 19	& 19	& 19	& 12	& 12	& 19 \\
    5	& 19	& 5	& 13	& 13	& 19	& 19	& 19	& 13	& 13	& 13	& 19 \\
    6	& 6	& 6	& 6	& 6	& 6	& 6	& 6	& 6	& 15	& 6	& 6 \\
    7	& 7	& 7	& 7	& 7	& 17	& 7	& 7	& 7	& 19	& 7	& 7 \\
    8	& 8	& 8	& 8	& 8	& 8	& 18	& 8	& 8	& 19	& 8	& 8 \\
    18	& 18	& 18	& 18	& 18	& 18	& 18	& 18	& 18	& 9	& 9	& 18 \\
    2	& 19	& 19	& 19	& 19	& 19	& 19	& 19	& 1	& 19	& 19	& 19 \\
    0	& 0	& 0	& 0	& 0	& 0	& 0	& 0	& 0	& 0	& 0	& 0 \\
    \bottomrule
\end{tabular}
\end{table}

\section{Complejidad}
Analizando el proceso utilizado por el DFA para determinar los distintos 
tokens que se van a estilizar utilizando \texttt{html} únicamente requiere 
analizar cada caracter del archivo leído una única vez, por lo que la 
complejidad de la solución planteada es $O(n)$, ya que siempre se debe de 
analizar todo el archivo por completo.

\section{Reflexión}
Una de las implementaciones éticas de esta tecnología de análisis sintáctico 
mediante autómatas determinísticos es el reconocimiento de sentimiento en un 
texto. Esto se puede aplicar a, por ejemplo, llamadas de emergencia de 911 
para determinar la urgencia de la llamada en caso de que no hayan agentes de 
respuesta disponibles, por lo que en lugar de poner la llamada en espera o en 
grabadora, se puede reaccionar y llenar el reporte mediante un reconocedor de 
grupos de palabras y así determinare is se necesita una ambulancia, una 
patrulla, o algun otro tipo de ayuda de emergencia.

\end{document}
